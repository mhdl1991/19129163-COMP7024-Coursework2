\documentclass{article}
\usepackage{graphicx} % Required for inserting images

\usepackage[
    backend=biber,
    style=authoryear,
    sorting=ynt
]{biblatex} % Required for adding references

\usepackage[inkscapeformat=png]{svg} 
\usepackage{appendix}
\usepackage{calc}
\usepackage{float}
\usepackage{fontspec}
\usepackage{pdfpages}
\addbibresource{bibliography.bib}

\title{COMP7024 Coursework2- Developing a file encryption system for the Minix Operating System}
\author{19129163 Mohammad Ali Khan}
\date{April 2023}

\begin{document}

\maketitle

\section{Introduction}
    \paragraph{}The objective of this project is to develop a program for the Minix 3.4 Operating System (OS) that will encrypt and decrypt files on the system.
    \paragraph{}This program will be written in the C programming language, and has a GitHub repository which can be found at https://github.com/mhdl1991/19129163-COMP7024-Coursework2
    \paragraph{}This program will be developed as a \textbf{daemon}.

\section{A Brief note on Daemons}
    \paragraph{}A \textbf{Daemon} (sometimes claimed to be an acronym for\textit{Disk And Execution MONitor}) is a computer program (usually in a Unix environment) that runs as a background process, rather than being under the direct control of a interactive user. Another terms for daemons are \textit{service}, \textit{ghost job}, or \textit{started task}
    \paragraph{}Examples of daemons include \textbf{init}, \textbf{crond}, \textbf{httpd}, and \textbf{syncd}, all of which perform useful tasks in the operating system.

\section{Requirements}
\paragraph{}We will need the following:
\begin{itemize}
    \item Knowledge of the Minix OS and filesystem
    \item libraries for performing Encrpytion and file handling (some may already be installed as part of the Minix OS)
    \item enough memory for handling the encryption and decryption
    \item Some way to store credentials (keys and initialization vectors)
    \item Some way to know which files have been encrypted and which have not
    \item permissions and policy settings to allow the daemon to alter files.
\end{itemize}

\section{Encryption}
\paragraph{}A common bit of advice regarding encryption is \textit{"Never roll your own cryptosystem"}- from technical and security standpoints it is better to use an existing, tried and tested cryptosystem than to develop your own- From a \textbf{technical} standpoint, it's very difficult to build your own cryptosystem and test it, and to make it \textbf{secure}, which ties into the \textbf{security} standpoint for not rolling your own cryptosystem.

\paragraph{}So for this system, we have the option of using \textbf{OpenSSL}, which comes with Minix 3, but an updated version can be installed using the pkgin utility, and supports a large number of encryption algorithms and ciphers.

\paragraph{}As per the OpenSSL documentation, it contains RSA, SHA, DES, SSL, TLS, and AES cipher suites/families. Care must be taken in the selection as it also currently includes a number of deprecated and older ciphers.

\paragraph{}For this program we are using 256-bit AES encryption in CBC (Cipher Block Chaining mode)

\section{Design}
\paragraph{}the program will have two main functions, \textit{file\_encrypt} and \textit{file\_decrypt}. Both functions will take a pointer to a file, a key, and an IV (initialization vector).


\section{Development}
\paragraph{}The program was first built as a standalone bit of C before attempting to integrate it with the Minix operating system. This was done to make sure the file encryption and decryption functions worked properly and didn't result in bugs, memory leaks, unintended alterations to files, or other unintended consequences, and to reduce damage to the Minix OS.



\section{Implementation}
\paragraph{}

\section{Testing}

\section{Conclusion}
\paragraph{}

\section{References}
\end{document}
