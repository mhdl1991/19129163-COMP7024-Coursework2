\documentclass{article}
\usepackage{graphicx} % Required for inserting images

\title{COMP7024 Coursework2- Developing a file encryption system for the Minix Operating System}
\author{19129163 Mohammad Ali Khan}
\date{April 2023}

\begin{document}

\maketitle

\section{Introduction}
\paragraph{}The objective of this project is to develop a program for Minix 3.4 that will encrypt and decrypt files on the system.
\paragraph{}The program will be written in the C programming language and be implemented as a \textit{device driver}.

\section{Requirements}
\paragraph{}Seeing as 

\section{Encryption}
\paragraph{}A common bit of advice regarding encryption that can be found in discussions pertaining to encryption is \textit{"Never roll your own cryptosystem"}- namely, from both a technical and security standpoints it is more advisable to use an existing, tried and tested cryptosystem than it is to make your own- From a \textbf{technical} standpoint, it's very difficult to build your own cryptosystem and test it, and to make it \textbf{secure}, which ties into the \textbf{security} standpoint for not rolling your own cryptosystem

\paragraph{}So for this system, we will be using \textbf{OpenSSL}, which comes with Minix 3, but an updated version can be installed using the pkgin utility, and has a  number of encryption algorithms and cipher families.

\paragraph{}As per the OpenSSL documentation, it contains RSA, SHA, DES, SSL, TLS cipher suites/families. Care must be taken in the selection as it also currently includes a number of deprecated ciphers


\section{Design}
\paragraph{}the program will have two main functions, \textit{file\_encrypt} and \textit{file\_decrypt}. 

\section{Implementation}
\paragraph{}As per the Minix documentation 

\section{References}
\end{document}
